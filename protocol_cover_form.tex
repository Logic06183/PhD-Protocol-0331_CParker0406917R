% Protocol Cover Page (LaTeX)

\begin{center}
    \Large\textbf{PhD Protocol Cover Page} \\
    \vspace{0.4cm}
\end{center}

\begin{tabular}{ll}
\textbf{Candidate's Surname:} & Parker \\
\textbf{First Name/s:} & Craig \\
\textbf{Student Number:} & 0406917R \\
\textbf{Current Qualifications:} & MM-PDM, Wits School of Governance; \\ 
& MSc Health Inequalities and Public Policy, University of Edinburgh; \\ 
& BS Neuroscience, King College, Tennessee USA \\
\textbf{Cell:} & 0792848593 \\
\textbf{E-mail:} & craig.parker@witsphr.org \\
\textbf{Degree:} & PhD in Public Health \\
\textbf{Part-time or Full-time:} & Part-time \\
\textbf{Department:} & School of Public Health \\
\textbf{Title of Proposed Research:} & Urban Heat Vulnerability and Health Outcomes in Johannesburg, South Africa \\
\textbf{Date:} & April 22, 2025 \\
\textbf{Ethics Reference:} & 220606 (Wits Human Research Ethics Committee) \\
\end{tabular}

\vspace{0.5cm}

\textbf{Supervisors:}

\begin{tabular}{ll}
\textbf{Supervisor 1:} & Dr. Admire Chikandiwa (Wits University) \\
\textbf{Percentage:} & 40\% \\
\textbf{Supervisor 2:} & Prof. Matthew Chersich (Trinity College/Wits University) \\
\textbf{Percentage:} & 25\% \\
\textbf{Supervisor 3:} & Prof. Akbar Waljee (University of Michigan) \\
\textbf{Percentage:} & 20\% \\
\end{tabular}

\vspace{0.5cm}

\textbf{Synopsis of Research:}

This research investigates urban heat vulnerability and health outcomes in Johannesburg, South Africa. Set against rising temperatures and climate change impacts, the study examines how urban form, socio-demographic characteristics, and climate factors interact to shape heat vulnerability patterns. The project employs innovative methodological approaches including Geographically Weighted Principal Component Analysis and causal machine learning to address spatial inequalities in heat exposure inherited from apartheid-era planning. Four interconnected objectives guide this work: (1) mapping heat vulnerability using geographically weighted approaches to account for spatial variations; (2) elucidating causal mechanisms linking urban characteristics and health outcomes; (3) developing predictive models for heat-related health impacts to inform early warning systems; and (4) translating findings into actionable policy recommendations. The research is framed within an infrastructural justice perspective, emphasizing how unequal access to protective infrastructure shapes vulnerability patterns. Expected outcomes include novel methodological approaches, an evidence-based adaptation framework, and targeted recommendations for municipal authorities. This work addresses critical knowledge gaps in understanding heat-health relationships in African urban contexts, with potential applications for climate resilience planning in similar urban environments globally.

% Add signature and other fields as needed
