\section*{Abstract}
\addcontentsline{toc}{section}{Abstract}
This research proposal investigates the complex relationship between urban heat and health in Johannesburg, South Africa. As climate change drives increasing temperatures globally, urban populations face heightened health risks, with vulnerable communities disproportionately affected. 
The study employs a multidimensional approach through three primary objectives: (1) mapping intra-urban heat vulnerability by integrating environmental, socio-economic, and health data; (2) delineating heat-health dynamics through a two-stage explanatory modeling approach that combines hypothesis generation with targeted testing to uncover physiological pathways and temporal effects; and (3) developing a stratified predictive model for heat-related health outcomes.
Drawing on clinical trial data, satellite imagery, climate records, and socioeconomic surveys, this research will apply advanced statistical and machine learning techniques to create vulnerability maps, explain complex relationships, and generate predictive models. The findings aim to inform targeted public health interventions, urban planning decisions, and climate adaptation strategies to protect vulnerable populations from increasing heat exposure in Johannesburg and potentially other African urban centers.

\noindent\rule{\textwidth}{0.5pt}

\vspace{0.5cm}
\noindent\textbf{Keywords:} urban heat, health outcomes, vulnerability mapping, machine learning, climate change, Johannesburg

\section*{Acknowledgements}
\addcontentsline{toc}{section}{Acknowledgements}

This research was conducted as part of the HEat and HEalth African Transdisciplinary (HE²AT) Center initiative, supported by the NIH Common Fund under Award Number U54 TW 012083.