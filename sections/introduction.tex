\section{Introduction and Background}

\subsection{Climate Change and Heat-Health Impacts in the Johannesburg Context}

Climate change has driven a global temperature increase of over 1.2°C since the Industrial Revolution, with African regions experiencing higher-than-average temperature increases \citep{IPCC2022}. Urban areas are particularly affected due to development patterns and land use changes, with high temperatures increasingly linked to mortality and morbidity, especially during heatwaves \citep{Gasparrini2015, Analitis2018}.

Heat-related risks are rising in Johannesburg at 1753 meters elevation with over 5.87 million inhabitants \citep{Worldometer2023}. The city recorded its highest temperature of 38°C in January 2016, breaking the previous record of 36.5°C from November 2015 \citep{Strydom2016}. Studies show that above a threshold of approximately 18.7°C apparent temperature, all-cause mortality rises by 0.9\% per 1°C increase and by 2.1\% per 1°C among seniors \citep{Wichmann2017, Scovronick2018}. While official records counted only 11 direct heat-related deaths in a recent assessment, researchers estimate the actual heat burden to be much higher \citep{Chersich2023}.

Climate models project substantial warming for Johannesburg. By late-century under a high-emissions scenario, mean annual temperatures in interior South Africa could rise 6--7°C above late 20th-century baselines \citep{Engelbrecht2015}. Even by 2050, Johannesburg may warm by approximately 2°C if global emissions remain high \citep{Engelbrecht2015, Souverijns2022}, with hot nights (minimum temperature >20°C) projected to quadruple from about 10 per year to approximately 40 per year -- and up to 100 in the city's most heat-prone neighbourhoods \citep{WorldBank2024}. Researchers estimate an additional 3--4 weeks of very hot days per year by mid-century \citep{Garland2015}, and the IPCC warns that beyond +2°C of global warming, heat-attributable mortality and morbidity in Africa will sharply rise \citep{IPCC2022}.

\subsection{Socio-Spatial Inequity and Heat Vulnerability}

Johannesburg's subtropical highland climate features hot summer days (October-April), often with afternoon thundershowers, and dry, sunny winter days with cold nights (May-September) \citep{Tyson2000}. However, the city's socio-spatial layout -- largely a legacy of apartheid-era planning -- significantly shapes heat vulnerability patterns today. Historically, policies created wealthy, low-density suburbs with ample green spaces alongside dense townships with minimal vegetation or services \citep{Giombini2022, Venter2020}. This has resulted in stark contrasts in urban heat exposure, with lush neighbourhoods enjoying cooler microclimates, while nearby townships can be approximately 6°C hotter than the surrounding countryside \citep{WorldBank2024, Habitat2023}.

Housing quality differences further exacerbate exposure: informal housing can become significantly hotter, with indoor temperatures up to 15°C higher than in modern housing during the day \citep{Naicker2017}. Apartheid geography has effectively embedded vulnerability into Johannesburg's landscape, clustering heat risk in marginalized communities \citep{Strauss2019}.

\subsection{Conceptual Framework and Research Gaps}

This research employs a comprehensive framework for heat vulnerability encompassing three interconnected dimensions: exposure, sensitivity, and adaptive capacity \citep{IPCC2022}:

\begin{itemize}
    \item \textit{Exposure} refers to heat stress degree and duration, with dense urban areas showing significantly higher surface temperatures (up to 5°C) compared to well-vegetated neighbourhoods \citep{Li2017, Santamouris2015}.
    
    \item \textit{Sensitivity} reflects population susceptibility influenced by socio-economic conditions and health status, with chronic conditions significantly increasing heat-related health risks \citep{Watts2023, Khosla2021, Souverijns2022}.
    
    \item \textit{Adaptive capacity} depends on access to healthcare, cooling infrastructure, and social support systems \citep{Ansah2024}, with limited access to healthcare affecting vulnerability to heat, as demonstrated by increased heat-related mortality in areas with restricted medical services \citep{Murage2020}.
\end{itemize}

Despite growing climate health research worldwide, significant gaps remain in understanding the dynamics of heat health in urban African contexts. Most existing studies focus on high-income regions, overlooking the distinct characteristics of African cities \citep{Khine2023, Pasquini2020}. Key limitations include:

\begin{enumerate}
    \item \textbf{Scarcity of African Urban Heat-Health Studies:} Limited research examining the impacts of heat on health in African urban settings hinders the development of region-specific interventions \citep{Ncongwane2021, Wright2019}.

    \item \textbf{Siloed Disciplinary Approaches:} Lack of interdisciplinary collaboration results in fragmented insights that fail to capture the multifaceted nature of heat-health challenges \citep{Jack}.

    \item \textbf{Unique Urban Challenges:} African cities face distinct issues stemming from historical development patterns, unique disease profiles, and diverse urban morphologies requiring tailored research approaches \citep{Giombini2022, Venter2020}.
\end{enumerate}

Johannesburg exemplifies these challenges through its historical development and urban disparities \citep{Strauss2019}, high disease burden from both communicable and non-communicable diseases \citep{Wright2021}, and accelerated warming projections \citep{Engelbrecht2015, Souverijns2022}. Addressing these gaps is crucial for developing heat health strategies tailored to African urban contexts.