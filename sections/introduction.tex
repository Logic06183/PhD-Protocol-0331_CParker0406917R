\section{Introduction and Background}

\subsection{Climate Change and Heat-Health Impacts in the Johannesburg Context}

Climate change has intensified heat-related health risks in Johannesburg, where rising temperatures are linked to increasing mortality and morbidity \citep{Gasparrini2015, Romanello2023}. With over 5.87 million inhabitants, the city faces substantial warming projections—by 2050, mean temperatures may rise approximately 2°C with hot nights projected to quadruple \citep{Engelbrecht2015, WorldBank2024}. Recent epidemiological evidence demonstrates the severity of this threat: above 18.7°C apparent temperature, all-cause mortality rises by 0.9\% per 1°C increase, with seniors experiencing 2.1\% increases \citep{Wichmann2017}. The last five years have seen a marked acceleration in heat-related mortality across global cities, with \citet{Armstrong2023} documenting a 68\% increase in heat-attributable deaths in urban areas since 2018. For Sub-Saharan African cities specifically, \citet{Gasparrini2022} found mortality impacts 2-3 times higher than global averages due to limited adaptive infrastructure. The IPCC warns that beyond +2°C of global warming, heat-attributable health impacts in Africa will sharply rise \citep{IPCC2024}, with urban populations facing disproportionate risks.

\subsection{Socio-Spatial Inequity and Heat Vulnerability}

Johannesburg's socio-spatial layout—largely a legacy of apartheid-era planning—significantly shapes contemporary heat vulnerability patterns. Historical policies created wealthy suburbs with green spaces alongside dense townships with minimal vegetation, resulting in temperature differentials of approximately 6°C between affluent neighborhoods and informal settlements \citep{WorldBank2024}. Housing quality further exacerbates this disparity, with informal dwellings experiencing up to 15°C higher indoor temperatures than formal housing \citep{Naicker2017}. This embedded vulnerability continues to cluster heat-health risks in historically marginalized communities \citep{Strauss2019}.

\subsection{Research Context and Positionality}
This research builds upon ongoing work at the HE²AT Center, which has established baseline heat vulnerability assessment frameworks for South African cities \citep{Jack}. As a researcher within this initiative since 2022, I have contributed to the center's vulnerability mapping work but identified critical gaps in existing approaches—specifically the need for locally-adaptive statistical methods and dynamic predictive capabilities. The Heat Center has successfully mapped static vulnerability patterns, but has not yet developed causal explanations or predictive applications. My position at the intersection of public health, climate science, and data analytics provides a unique perspective to extend these frameworks through advanced statistical techniques while maintaining focus on local context and community relevance.

My background in both quantitative methods and participatory research informs a multidisciplinary approach that translates complex data into meaningful insights for health planning. I acknowledge that my position as an academic researcher shapes how I interpret vulnerability patterns and I recognize the value of community perspectives throughout the analytical process for meaningful interpretation of results. This work specifically addresses gaps in the Heat Center's existing vulnerability assessments by developing dynamic predictive models that can account for temporal changes in vulnerability patterns across Johannesburg's diverse socioeconomic landscape, contributing methodological innovations while building on the center's established knowledge base.

\subsection{Conceptual Framework and Research Gaps}

This research employs a comprehensive framework for heat vulnerability encompassing three interconnected dimensions: exposure, sensitivity, and adaptive capacity \citep{IPCC2022}. The conceptual framework presented in Figure \ref{fig:conceptual_framework} (see Section 2) illustrates these relationships and their determinants within Johannesburg's urban context.

These components interact as follows:
\begin{itemize}
    \item \textit{Exposure} refers to heat stress degree and duration, with dense urban areas showing significantly higher surface temperatures (up to 5\textdegree C) compared to well-vegetated neighbourhoods \citep{Li2017, Santamouris2015}.
    
    \item \textit{Sensitivity} reflects population susceptibility influenced by socio-economic conditions and health status, with chronic conditions significantly increasing heat-related health risks \citep{Watts2023, Khosla2021, Souverijns2022}.
    
    \item \textit{Adaptive capacity} depends on access to healthcare, cooling infrastructure, and social support systems \citep{Ansah2024}, with limited access to healthcare affecting vulnerability to heat, as demonstrated by increased heat-related mortality in areas with restricted medical services \citep{Murage2020}.
\end{itemize}

\subsection{Recent Evidence and Research Gaps}
Recent studies have strengthened our understanding of heat-mortality relationships in African urban contexts. Parker et al. \citep{Parker2023} demonstrated that in Johannesburg, heat vulnerability clusters in historically disadvantaged areas, with environmental exposure explaining 31.5\% of variance. The Lancet Countdown \citep{Romanello2023} reports escalating impacts, with an estimated 5 million people globally dying annually from suboptimal temperatures. Despite this growing evidence, significant research gaps persist in African urban contexts \citep{Khine2023}, including: (1) scarcity of region-specific studies; (2) siloed disciplinary approaches failing to capture multifaceted heat-health relationships; and (3) limited recognition of unique urban challenges stemming from historical development patterns and disease profiles. Johannesburg exemplifies these challenges through its urban disparities and accelerated warming projections \citep{Engelbrecht2015}.