\section{Expected Outcomes and Impact}

\subsection{Research Outputs}
This doctoral research will generate several significant outputs, including a spatially explicit heat vulnerability index for Johannesburg, a comprehensive explanatory model of urban heat-health relationships, and a validated predictive framework for heat-health outcomes under varied climate scenarios. The work will also produce evidence-based policy recommendations for city planning and public health systems, culminating in at least three peer-reviewed publications addressing vulnerability mapping, explanatory modelling, and predictive applications.

\subsection{Anticipated Impact and Infrastructural Justice Framing}
The research is expected to yield significant impacts by enhancing understanding of fine-scale spatial patterns of heat vulnerability in Johannesburg, improving heat-health early warning systems, and identifying priority areas for targeted intervention. This will enable more efficient resource allocation and provide robust, evidence-based policy guidance for urban heat adaptation, particularly for informal settlements and densely populated areas. These methodological advances in integrated urban heat health research can be adapted for other African urban contexts.

\subsubsection{Climate-Resilient Cities and Infrastructural Justice}
Beyond its analytical contributions, this research serves as a platform for advancing infrastructural justice in the context of urban climate resilience. By explicitly mapping the distribution of heat vulnerability in relation to historical urban development patterns, the work highlights how infrastructure inequities—in green spaces, cooling resources, healthcare access, and housing quality—create differential climate risk exposure along socioeconomic and racial lines. Recent findings from the 2023 Lancet Countdown on Health and Climate Change emphasize that these infrastructure disparities function as determinants of health that amplify climate risks in precisely the communities with least historical responsibility for emissions \citep{Romanello2023}. This framing recognizes that vulnerability to heat is not merely a function of temperature but a manifestation of systemic infrastructure decisions that have unequally distributed adaptive capacity across the urban landscape.

The project's findings will inform a justice-oriented approach to infrastructure development that prioritizes addressing historical imbalances in resource allocation, building on methodological advances in identifying causal pathways in environmental justice contexts \citep{Velasquez2023}. Specifically, results will identify areas where structural interventions (green infrastructure expansion, cooling centers, healthcare facility development) are most urgently needed to reduce disproportionate vulnerability as emphasized in the latest IPCC synthesis report \citep{IPCC2024}. This focus on infrastructural justice extends beyond technical solutions to incorporate community governance structures and local knowledge systems in designing context-appropriate adaptations. By framing climate resilience through an equity lens, this research contributes to transformative urban planning that addresses current vulnerability while preventing the perpetuation of historical injustices in future climate adaptation efforts—an approach increasingly recognized as essential for sustainable urban development in the context of accelerating climate change \citep{Praharaj2024}.

\subsection{Knowledge Translation}
Research findings will be strategically disseminated through peer-reviewed publications in journals spanning public health, climate science, and urban planning disciplines. Additional dissemination channels include presentations at national and international conferences, targeted policy briefs for local and national authorities, community engagement workshops in vulnerable areas, and interactive data visualization tools to facilitate stakeholder decision-making. Detailed impact assessment metrics and comprehensive dissemination strategies are provided in Appendix D.