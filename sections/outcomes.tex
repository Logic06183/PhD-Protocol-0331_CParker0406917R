\section{Expected Outcomes and Impact}

\subsection{Research Outputs}
This doctoral research will generate several significant outputs, including a spatially explicit heat vulnerability index for Johannesburg, a comprehensive explanatory model of urban heat-health relationships, and a validated predictive framework for heat-health outcomes under varied climate scenarios. These outputs will provide a foundation that could inform future policy development, though the actual formulation of specific interventions falls outside the scope of this PhD. The work will culminate in at least three peer-reviewed publications addressing vulnerability mapping, explanatory modelling, and predictive applications.

\subsection{Anticipated Impact and Infrastructural Justice Framing}
This research will generate both scholarly and applied outcomes, advancing methodology while informing practical adaptation solutions. The primary outcomes include:

\begin{enumerate}
    \item \textbf{Novel Methodological Approaches:} Integrating geographically weighted principal component analysis with causal machine learning for contexts with high socio-spatial inequality.
    
    \item \textbf{Evidence-Based Knowledge Framework:} A comprehensive understanding of spatial heat vulnerability patterns and causal mechanisms that could inform future adaptation strategies.
    
    \item \textbf{Predictive Modeling Capabilities:} Robust predictive models that demonstrate the potential health impacts of extreme heat events across different urban contexts.
\end{enumerate}

While this research will generate valuable insights that could serve as a foundation for future policy development, it is important to note that the actual design and implementation of specific interventions is beyond the scope of this PhD project.

This research advances an infrastructural justice perspective on heat vulnerability \citep{Romanello2023}, examining how historical development patterns shape contemporary risk distributions. This framing positions heat vulnerability within socio-political processes that determine whose communities receive protection from climate hazards \citep{IPCC2024} and establishes a foundation for building climate-resilient urban environments.

Infrastructural justice, as applied in this research, recognizes that vulnerability to heat is not merely a function of temperature but a manifestation of systemic infrastructure decisions that have unequally distributed adaptive capacity across the urban landscape. The geographically weighted approaches employed are specifically designed to identify areas where mixed socioeconomic conditions create complex vulnerability patterns—patterns often missed by conventional analysis methods that aggregate at administrative boundaries.

The project's findings will provide a robust knowledge foundation for understanding the spatial and temporal distribution of urban heat vulnerability through time (2000-2022), capturing how vulnerability has evolved over two decades of urban transformation and climate change. By implementing causal machine learning methods, this study will identify specific leverage points for adaptation planning within Johannesburg's diverse urban fabric \citep{Velasquez2023}. While the research will identify areas of disproportionate heat vulnerability as emphasized in the latest IPCC synthesis report \citep{IPCC2024}, the actual development and implementation of interventions falls outside the scope of this PhD project.

This focus on infrastructural justice provides both an analytical framework for understanding existing vulnerability patterns and a platform for conceptualizing climate-resilient urban futures. By conducting this analysis through an equity lens with explicit attention to historical development processes, this research contributes valuable knowledge that could inform future urban planning approaches toward more equitable and resilient cities—an analytical perspective increasingly recognized as essential for understanding urban vulnerability in the context of accelerating climate change \citep{Praharaj2024}.

\subsection{Knowledge Translation}
Research findings will be primarily disseminated through peer-reviewed publications in journals spanning public health, climate science, and urban planning disciplines. Additional academic dissemination channels will include presentations at national and international conferences. While the development of specific interventions is beyond the scope of this PhD, the research outputs will be structured to serve as a foundation for future applied work by other researchers and policymakers. Detailed academic publication plans are provided in Appendix D.