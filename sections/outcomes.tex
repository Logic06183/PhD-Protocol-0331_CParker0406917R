\section{Expected Outcomes and Impact}

\subsection{Research Outputs}
This doctoral research will generate several significant outputs, including a spatially explicit heat vulnerability index for Johannesburg, a comprehensive explanatory model of urban heat-health relationships, and a validated predictive framework for heat-health outcomes under varied climate scenarios. The work will also produce evidence-based policy recommendations for city planning and public health systems, culminating in at least three peer-reviewed publications addressing vulnerability mapping, explanatory modelling, and predictive applications.

\subsection{Anticipated Impact}
The research is expected to yield significant impacts by enhancing understanding of fine-scale spatial patterns of heat vulnerability in Johannesburg, improving heat-health early warning systems, and identifying priority areas for targeted intervention. This will enable more efficient resource allocation and provide robust, evidence-based policy guidance for urban heat adaptation, particularly for informal settlements and densely populated areas. These methodological advances in integrated urban heat health research can be adapted for other African urban contexts.

\subsection{Knowledge Translation}
Research findings will be strategically disseminated through peer-reviewed publications in journals spanning public health, climate science, and urban planning disciplines. Additional dissemination channels include presentations at national and international conferences, targeted policy briefs for local and national authorities, community engagement workshops in vulnerable areas, and interactive data visualization tools to facilitate stakeholder decision-making. Detailed impact assessment metrics and comprehensive dissemination strategies are provided in Appendix D.