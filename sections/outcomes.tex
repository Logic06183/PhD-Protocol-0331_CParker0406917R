\section{Expected Outcomes and Impact}

\subsection{Research Outputs}
This doctoral research will generate several significant outputs, including a spatially explicit heat vulnerability index for Johannesburg, a comprehensive explanatory model of urban heat-health relationships, and a validated predictive framework for heat-health outcomes under varied climate scenarios. The work will also produce evidence-based policy recommendations for city planning and public health systems, culminating in at least three peer-reviewed publications addressing vulnerability mapping, explanatory modelling, and predictive applications.

\subsection{Anticipated Impact and Infrastructural Justice Framing}
This research will generate both scholarly and applied outcomes, advancing methodology while informing practical adaptation solutions. The primary outcomes include:

\begin{enumerate}
    \item \textbf{Novel Methodological Approaches:} Integrating geographically weighted principal component analysis with causal machine learning for contexts with high socio-spatial inequality.
    
    \item \textbf{Evidence-Based Adaptation Framework:} A prioritization system for heat interventions based on vulnerability reduction potential and implementation feasibility.
    
    \item \textbf{Policy Translation:} Targeted recommendations for municipal authorities and health systems addressing both emergency measures and infrastructural solutions.
\end{enumerate}

This research advances an infrastructural justice perspective on heat vulnerability \citep{Romanello2023}, examining how historical development patterns shape contemporary risk distributions. This framing positions heat vulnerability within socio-political processes that determine whose communities receive protection from climate hazards \citep{IPCC2024}.

Research outputs will be made available through open access repositories and disseminated via academic publications, policy briefs, and community materials in local languages. This framing recognizes that vulnerability to heat is not merely a function of temperature but a manifestation of systemic infrastructure decisions that have unequally distributed adaptive capacity across the urban landscape.

The project's findings will inform a justice-oriented approach to infrastructure development that prioritizes addressing historical imbalances in resource allocation, building on methodological advances in identifying causal pathways in environmental justice contexts \citep{Velasquez2023}. Specifically, results will identify areas where structural interventions (green infrastructure expansion, cooling centers, healthcare facility development) are most urgently needed to reduce disproportionate vulnerability as emphasized in the latest IPCC synthesis report \citep{IPCC2024}. This focus on infrastructural justice extends beyond technical solutions to incorporate community governance structures and local knowledge systems in designing context-appropriate adaptations. By framing climate resilience through an equity lens, this research contributes to transformative urban planning that addresses current vulnerability while preventing the perpetuation of historical injustices in future climate adaptation efforts—an approach increasingly recognized as essential for sustainable urban development in the context of accelerating climate change \citep{Praharaj2024}.

\subsection{Knowledge Translation}
Research findings will be strategically disseminated through peer-reviewed publications in journals spanning public health, climate science, and urban planning disciplines. Additional dissemination channels include presentations at national and international conferences, targeted policy briefs for local and national authorities, community engagement workshops in vulnerable areas, and interactive data visualization tools to facilitate stakeholder decision-making. Detailed impact assessment metrics and comprehensive dissemination strategies are provided in Appendix D.